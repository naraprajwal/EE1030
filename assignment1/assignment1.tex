%iffalse
\let\negmedspace\undefined
\let\negthickspace\undefined
\documentclass[journal,12pt,onecolumn]{IEEEtran}
\usepackage{cite}
\usepackage{amsmath,amssymb,amsfonts,amsthm}
\usepackage{algorithmic}
\usepackage{graphicx}
\usepackage{textcomp}
\usepackage{xcolor}
\usepackage{txfonts}
\usepackage{listings}
\usepackage{enumitem}
\usepackage{mathtools}
\usepackage{gensymb}
\usepackage{comment}
\usepackage[breaklinks=true]{hyperref}
\usepackage{tkz-euclide} 
\usepackage{listings}
\usepackage{gvv}                                        
%\def\inputGnumericTable{}                                 
\usepackage[latin1]{inputenc}                                
\usepackage{color}                                            
\usepackage{array}                                            
\usepackage{longtable}                                       
\usepackage{calc}                                             
\usepackage{multirow}                                         
\usepackage{hhline}                                           
\usepackage{ifthen}                                           
\usepackage{lscape}
\usepackage{tabularx}
\usepackage{array}
\usepackage{float}


\newtheorem{theorem}{Theorem}[section]
\newtheorem{problem}{Problem}
\newtheorem{proposition}{Proposition}[section]
\newtheorem{lemma}{Lemma}[section]
\newtheorem{corollary}[theorem]{Corollary}
\newtheorem{example}{Example}[section]
\newtheorem{definition}[problem]{Definition}
\newcommand{\BEQA}{\begin{eqnarray}}
\newcommand{\EEQA}{\end{eqnarray}}
\newcommand{\define}{\stackrel{\triangle}{=}}
\theoremstyle{remark}
\newtheorem{rem}{Remark}

% Marks the beginning of the document
\begin{document}
\bibliographystyle{IEEEtran}
\vspace{3cm}

\title{Chapter-11-
        Limits,Continuity and  Differentiability}
\author{EE24BTECH11051 - Prajwal}
\maketitle
\newpage
\bigskip

\renewcommand{\thefigure}{\theenumi}
\renewcommand{\thetable}{\theenumi}


\section{Subjective Problems}
   


\begin{enumerate}

\item Let $f(x)$ be a function satisfying the condition $f(-x)=f(x)$ for real $x$.If $f'(0)$ exists, find its  value.\hfill (1987 - 2 Marks)
\item Find the values of $a$ and $b$ so that the function\\
$f(x) = $
$\begin{cases}
    x + a{\sqrt2}\sin{x} & \text{if}\  0 \leq x < 4  \\
    2x \cot{x} + b & \text{if} \  \frac{\pi}{4}\leq x \leq  \frac{\pi}{2}\\
    a \cos{2x} - b \sin{x} & \text{if}\ \frac{\pi}{2} < x \leq \pi
\end{cases}$\\ 
is continuous for $0\leq x \leq \pi $.\hfill (1989 - 2 Marks)
\item Draw a graph of the function $y = [x]+|1-x|, -1 \leq x \leq 3$. Determine the points, if any, where this function is not differentiable.\hfill (1989 - 4 Marks)
\item Let $f(x) =$
$\begin{cases}
        \frac{1-\cos{4x}}{x^2} & \text{if}\ x < 0\\
    a & \text{if}\ x = 0 \\
    \frac{\sqrt{x}}{\sqrt{16+\sqrt{x}}- 4} & \text{if}\ x > 0 
\end{cases}$\\
Determine the value of $a$, if possible, so that the function is continuous at $x=0$. \hfill (1990 - 4 Marks)
\item A function $f : R \rightarrow R$ satisfies the equation $f(x+y) = f(x)f(y)$ for all $x,y$ in $R$ and $f(x) \neq 0 $ for any $x$ in $R$. Let the function be differentiable at $x = 0$ and $f'(0)=2.$ Show that $f'(x) = 2f(x)$ for all $x$ in $R$. Hence, determine $f(x)$.\hfil (1990 - 4 Marks)
\item Find $\lim_{x \to 0} \{\tan(\frac{\pi}{4} + x)\}^{\frac{1}{x}}$
\hfill (1993 - 2 Marks)
\item Let $f(x) =$
$\begin{cases}
	\{1 + \abs{\sin{x}}\}^\frac{a}{\abs{\sin{x}}} & \text{if}\ - \frac{\pi}{6} < x < 0 \\
    b & \text{if}\ x = 0\\
    e^\frac{\tan{2x}}{\tan{3x}} & \text{if}\ 0 < x < \frac{\pi}{6}\\
    
\end{cases}$\\
Determine $a$ and $b$ such that $f(x)$ is continuous at $x=0$\hfill (1994 - 4 Marks)
\item  Let $f\brak{\frac{x+y}{2}} = \frac{f(x)+f(y)}{2}$ for real $x$ and $y$. If  $f'(0)$ exists and equals $-1$ and $f(0)=1$, find $f(2)$.\hfill (1995 - 5 Marks)
\item Determine the values of $x$ for which the following function fails to be continuous or diffferentiable:\\
$f(x) =$
$\begin{cases}
    1 - x, & \text{if}\ x<1\\
    (1-x)(2-x), & \text{if}\ 1 \leq x \leq 2\\
    3-x, & \text{if}\ x>2
\end{cases}$\\
Justify your answer.\hfill (1997 - 5 Marks)
\item Let $f(x),x\geq 0$, be a non-negative continuous function, and let $F(x) = \int_{0}^{x} f(t) \, dt, x \geq 0$. If for some $c>0,f(x) \leq cF(x)$ for all $x \geq 0$, then show that $f(x)=0$ for all $x \geq 0$.\hfill (2001 - 5 Marks)
\item Let $\alpha \in R$.Prove that a function $f;R\rightarrow R$ is differentiable at $\alpha$ if and only if there is a function $g:R \rightarrow R$ is differentiable at continuous at $\alpha$ and satisfies $f(x)-f(\alpha)=g(x)(x-\alpha)$ for all $x \in R$.\hfill (1995 - 5 Marks)
\item Let $f(x) = $ 
$\begin{cases}
    x + a & \text{if}\ x < 0\\
    |x + a| & \text{if}\ x \geq 0
\end{cases}$
and 
$g(x) = $
$\begin{cases}
    x + 1 & \text{if}\ x < 0\\
    (x - 1)^2 + b & \text{if}\ x \ geq 0,\\
     \end{cases}$
      where $a$ and $b$ are non-negative real numbers. Determine the composite function $(gof)(x)$ is continuous for all real $x$, determine the values of $a$ and $b$. Further, for these values of $a$ and $b$, is $gof$ is differentiable at $x = 0$? Justify your answer.\hfill (2002 - 5 Marks)
\item If a function $[-2a , 2a]\rightarrow R$ is an function such that $f(x) = f(2a - x)$ for $x \in [a , 2a]$ and the left hand derivatives at $x = a$ is $0$ then find the left hand derivative at $x = -a$.
	\hfill (2003 - 2 Marks)
\item $f'(0) = \lim_{n \to \infty} nf(\frac{1}{n})$ and $f(0) = 0$. Using this find\\
	$\lim_{x \to \infty}\ \brak{(n+1)\frac{2}{\pi}\cos^{-1}(\frac{1}{n})-n}\ , \abs{\cos^{-1}\frac{1}{n}}<\frac{\pi}{2}$\\  
\hfill (2004 - 2 Marks)
\item If $\abs{c} \leq \frac{1}{2}$ and $f(x)$ is a diiferentiable function at $x = 0$ given by\\
$f(x) = $
$\begin{cases}
	bsin^{-1} \brak{\frac{c + x}{2}} & \text{if}\ -\frac{1}{2}<x<0\\
    \frac{1}{2} & \text{if}\ x=0\\
    \frac{e^\frac{ax}{2} - 1}{x} & \text{if}\ 0<x<\frac{1}{2}
\end{cases}$\\
Find the value of $'a'$ and prove that $b^2 = 4 - c^2$.
\hfill (2004 - 4 Marks)






      
\end{enumerate}

\end{document}



























            
