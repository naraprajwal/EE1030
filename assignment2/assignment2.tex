%iffalse
\let\negmedspace\undefined
\let\negthickspace\undefined
\documentclass[journal,12pt,twocolumn]{IEEEtran}
\usepackage{cite}
\usepackage{amsmath,amssymb,amsfonts,amsthm}
\usepackage{algorithmic}
\usepackage{graphicx}
\usepackage{textcomp}
\usepackage{xcolor}
\usepackage{txfonts}
\usepackage{listings}
\usepackage{enumitem}
\usepackage{mathtools}
\usepackage{gensymb}
\usepackage{comment}
\usepackage[breaklinks=true]{hyperref}
\usepackage{tkz-euclide} 
\usepackage{listings}
\usepackage{gvv}                                        
%\def\inputGnumericTable{}                                 
\usepackage[latin1]{inputenc}                                
\usepackage{color}                                            
\usepackage{array}                                            
\usepackage{longtable}                                       
\usepackage{calc}                                             
\usepackage{multirow}                                         
\usepackage{hhline}                                           
\usepackage{ifthen}                                           
\usepackage{lscape}
\usepackage{tabularx}
\usepackage{array}
\usepackage{float}


\newtheorem{theorem}{Theorem}[section]
\newtheorem{problem}{Problem}
\newtheorem{proposition}{Proposition}[section]
\newtheorem{lemma}{Lemma}[section]
\newtheorem{corollary}[theorem]{Corollary}
\newtheorem{example}{Example}[section]
\newtheorem{definition}[problem]{Definition}
\newcommand{\BEQA}{\begin{eqnarray}}
\newcommand{\EEQA}{\end{eqnarray}}
\newcommand{\define}{\stackrel{\triangle}{=}}
\theoremstyle{remark}
\newtheorem{rem}{Remark}

% Marks the beginning of the document
\begin{document}
\bibliographystyle{IEEEtran}
\vspace{3cm}

\title{Properties of Triangle}
\author{ee24btech11051 - Prajwal}
\maketitle
\newpage
\bigskip

\renewcommand{\thefigure}{\theenumi}
\renewcommand{\thetable}{\theenumi}
\section{Fill in the Blanks}
\begin{enumerate}
    \item In a $\triangle ABC$, $\angle A=90\degree$ and $AD$ is an altitude. Complete the relation\\
    \\
    $\displaystyle \frac{BD}{BA} = \displaystyle \frac{AB}{(\dots)}$
    \hfill (1980)
    
    \item $ABC$ is a triangle, $P$ is a point on $AB$, and $Q$ is point on $AC$ such that $\angle AQP = \angle ABC$. Complete the relation
    $\displaystyle\frac{area\ of \triangle APQ}{area\ of \triangle ABC} =\displaystyle\frac{(\dots)}{AC^2}$
    \hfill (1980)
    
    \item $ABC$ is a triangle with $\angle B $ greater than $\angle C$ 
    $D$ and $E$ are the points on BC such that AD is perpendicular to BC and AE is the bisector of angle A .Complete the relation\\
    $\angle DAE = \displaystyle\frac{1}{2} [( ) - \angle C]$
    \hfill (1980)
    \item the set of all real numbers $a$ such that $a^2 + 2a, 2a + 3$ and $a^2 + 3a + 8$ are the sides of a triangle is \dots
    \hfill (1985 - 2 Marks)
    \item In a triangle $ABC$, if $\cot A$,$\cot B$,$\cot C$ are in A.P. ,then $a^2$,$b^2$,$c^2$,are in \dots progression \hfill (1985 - 2 Marks)
    \item A polygon of nine sides, each of length 2, is inscribed in a circle. The radius of the circle is \dots \hfill (1987 - 2 Marks) 
    \item If the angles of a triangle are $30\degree$ and $45\degree$ and the included side is $(\sqrt{3} + 1)$ cms, then the area of the triangle is \dots \hfill (1988 - 2 Marks)
    \item If the triangle $ABC$, $\displaystyle\frac{2\cos A}{a} + \displaystyle\frac{2\cos B}{b} + \displaystyle\frac{2\cos C}{c} = \displaystyle\frac{a}{bc} + \displaystyle\frac{b}{ac}$, then the value of the angle $A$ is \dots degrees. \hfil (1993 - 2 Marks)
    \item In the triangle $ABC$, $AD$ is the altitude from $A$. Given $b>c$, $\angle C=23 \degree$ and $AD = \displaystyle\frac{abc}{b^2 - c^2}$ then $\angle B = $ \dots \hfill (1994 - 2 Marks)
    \item A circle is inscribed in a equilateral triangle of a side a. The area of any square inscribed in this circle is \dots \hfill (1994 - 2 Marks)  
    \item In a triangle $ABC$, $a:b:c = 4:5:6$. The ratio of the radius of the circumstances to that of the incircle is \dots \hfill (1996 - 1 Marks) 
\end{enumerate}
\section{MCQ with one correct answer}
\begin{enumerate}
    \item If the bisector of the angle P of a triangle PQR meets QR in S, then
    \begin{enumerate}[label = (\alph*)]
        \item $QS = SR$
        \item $QS : SR = PR : PQ$
        \item $QS : SR = PQ : PR$
        \item None of these \hfill (1979)
    \end{enumerate}
    \item From the top of a light-house 60 meter high with its base at the sea level the angle of depression of a boat is  $15\degree$. The distance of the boat from the foot of the light house.
    \begin{enumerate}[label = (\alph*)]
    \item $\brak{\frac{\sqrt{3}-1}{\sqrt{3}+1}}60\ metres $
    \item $\brak{\frac{\sqrt{3}+1}{\sqrt{3}-1}}60\ metres $
    \item $\brak{\frac{\sqrt{3}+1}{\sqrt{3}-1}}^2 60\ metres $
    \item none of these \hfill (1983 - 2 Marks)
\end{enumerate}
    \item In the triangle $ABC$, angle A is the greater than angle B. If the measures of the angles A and B satisfies the equation $3\sin x - 4 \sin^3 x - k = 0, 0<k<1$ , then the measure of the angle C is 
     \begin{enumerate}[label = (\alph*)]
     \item $\frac{\pi}{3}$
     \item $\frac{\pi}{2}$
     \item $\frac{2\pi}{3}$
     \item $\frac{5\pi}{6}$ \hfill (1990 - 2 Marks)
\end{enumerate}
    \item If the lengths of the sides of triangles are 3,5,7 then the largest angles of the triangle is
    \begin{enumerate}[label = (\alph*)]
     \item $\frac{\pi}{2}$
     \item $\frac{5\pi}{6}$
     \item $\frac{2\pi}{3}$
     \item $\frac{3\pi}{4}$ \hfill (1994)
\end{enumerate}

\end{enumerate}

\end{document}