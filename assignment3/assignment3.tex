%iffalse
\let\negmedspace\undefined
\let\negthickspace\undefined
\documentclass[journal,12pt,twocolumn]{IEEEtran}
\usepackage{cite}
\usepackage{amsmath,amssymb,amsfonts,amsthm}
\usepackage{algorithmic}
\usepackage{graphicx}
\usepackage{textcomp}
\usepackage{xcolor}
\usepackage{txfonts}
\usepackage{listings}
\usepackage{enumitem}
\usepackage{mathtools}
\usepackage{gensymb}
\usepackage{comment}
\usepackage[breaklinks=true]{hyperref}
\usepackage{tkz-euclide} 
\usepackage{listings}
\usepackage{gvv}                                        
%\def\inputGnumericTable{}                                 
\usepackage[latin1]{inputenc}                                
\usepackage{color}
\usepackage{array}                                        
\usepackage{longtable}                                    
\usepackage{calc}                                         
\usepackage{multirow}                                     
\usepackage{hhline}                                       
\usepackage{ifthen}                                       
\usepackage{lscape}
\usepackage{tabularx}
\usepackage{array}
\usepackage{float}
\usepackage{enumitem,multicol}

\newtheorem{theorem}{Theorem}[section]
\newtheorem{problem}{Problem}
\newtheorem{proposition}{Proposition}[section]
\newtheorem{lemma}{Lemma}[section]
\newtheorem{corollary}[theorem]{Corollary}
\newtheorem{example}{Example}[section]
\newtheorem{definition}[problem]{Definition}
\newcommand{\BEQA}{\begin{eqnarray}}
\newcommand{\EEQA}{\end{eqnarray}}
\newcommand{\define}{\stackrel{\triangle}{=}}
\theoremstyle{remark}
\newtheorem{rem}{Remark}

% Marks the beginning of the document
\begin{document}
\bibliographystyle{IEEEtran}
\vspace{3cm}

\title{Chapter-20-
        Vector Algebra and Three Dimensional Geometry}
\author{EE24BTECH11051 - Prajwal}
\maketitle
\newpage
\bigskip

\renewcommand{\thefigure}{\theenumi}
\renewcommand{\thetable}{\theenumi}


\section{MCQs and One Correct Answer}
   


\begin{enumerate}
  \item Let $\vec{u},\vec{v}$ and $\vec{w}$ be vectors such that $\vec{u}+\vec{v}+\vec{w} = 0$. If $\abs{\vec{u}}=3$,$\abs{\vec{v}}=4$ and $\abs{\vec{w}}=5$ , then $\vec{u}.\vec{v}+\vec{v}.\vec{w}+\vec{u}.\vec{w}$ is
  \hfill (1995S)
 \begin{enumerate}
  \begin{multicols}{2}
  \item 47
  \item -25
  \item 0
  \item 25
  \end{multicols}
 \end{enumerate}   	
  \item If $\vec{a},\vec{b}$ and $\vec{c}$ are three non-coplanar vectors then 
  $\brak{\vec{a}+\vec{b}+\vec{c}}.\sbrak{\brak{\vec{a}+\vec{b}} \times \brak{\vec{a}+\vec{c}}}$ equals
  \hfill (1995S)
  \begin{enumerate}
  \begin{multicols}{2}
    \item $0$
    \item $\sbrak{\vec{a}\ \vec{b}\ \vec{c}}$ 
    \item $2\sbrak{\vec{a}\ \vec{b}\ \vec{c}}$ 
   \item $-\sbrak{\vec{a}\ \vec{b}\ \vec{c}}$
  \end{multicols}
  \end{enumerate}
\item Let $\vec{a}=2\vec{i}+\vec{j}-2\vec{k}$ and $\vec{b}=\vec{i}+\vec{j}$. If c is  a vector such that $\vec{a}.\vec{c}=\abs{\vec{c}}$,$\abs{\vec{c}-\vec{a}}=2\sqrt{2}$ and the angle between $\brak{\vec{a} \times \vec{b}}$ and $\vec{c}$ is $30\degree$, then $\abs{\brak{\vec{a} \times \vec{b}} \times \vec{c}} = $  
\hfill (1999 - 2 Marks)
\begin{enumerate}
\begin{multicols}{2}
    \item $\frac{2}{3}$
    \item $\frac{3}{2}$
    \item $2$
    \item $3$
\end{multicols}
\end{enumerate}
\item Let $\vec{a}=2\vec{i}+\vec{j}+\vec{k},\vec{b}=\vec{i}+2\vec{j}-\vec{k}$ and a unit vector c be coplanar. If $\vec{c}$ is perpendicular ti $\vec{a}$, then $\vec{c}=$
\hfill (1999 - 2 Marks)
\begin{enumerate}
\begin{multicols}{2}
\item $\frac{1}{\sqrt{2}}\brak{-\vec{j}+\vec{k}}$
\item $\frac{1}{\sqrt{3}}\brak{-\vec{i}-\vec{j}-\vec{k}}$ \item $\frac{1}{\sqrt{5}}\brak{\vec{i}-2\vec{j}}$    
\item $\frac{1}{\sqrt{3}}\brak{\vec{i}-\vec{j}-\vec{k}}$  \end{multicols}
\end{enumerate}
\item If the vectors $\vec{a},\vec{b}$ and $\vec{c}$ from the sides $BC,CA$ and $AB$ respectively of a triangle $ABC$, then
\hfill (2000S)
\begin{enumerate}
\begin{multicols}{2}
\item $\vec{a}.\vec{b}+\vec{b}.\vec{c}+\vec{c}.\vec{a} = 0$
\item $\vec{a}.\vec{b}=\vec{b}.\vec{c}=\vec{c}.\vec{a}$
\item $\vec{a} \times \vec{b}=\vec{b} \times \vec{c}=\vec{c} \times \vec{a}$
\item $\vec{a} \times \vec{b}+\vec{b} \times \vec{c}+\vec{c} \times \vec{a}=0$
\end{multicols}
\end{enumerate}
\item Let the vectors $\vec{a},\vec{b},\vec{c}$ and $\vec{d}$ be such that $\brak{\vec{a} \times \vec{b}} \times \brak{\vec{c} \times \vec{d}} = 0$. Let $A$ and $B$ be planes determined by the pairs of vectors $\vec{a},\vec{b}$ and $\vec{c},\vec{d}$ respectively. Then the angle between $A$ and $B$ is 
\hfill (2000S)
\begin{enumerate}
\begin{multicols}{4}
 \item $0$
 \item $\frac{\pi}{4}$
 \item $\frac{\pi}{3}$
 \item $\frac{\pi}{2}$
\end{multicols}
\end{enumerate}
\item If $\vec{a},\vec{b}$ and $\vec{c}$ are unit coplanar vectors, then the scalar triple product $\sbrak{2\vec{a}-\vec{b},2\vec{b}-\vec{c},2\vec{c}-\vec{a}} = $
\hfill (2000S)
\begin{enumerate}
\begin{multicols}{4}
    \item $0$
    \item $1$
    \item $-\sqrt{3}$
    \item $\sqrt{3}$
\end{multicols}
\end{enumerate}
\item Let $\vec{a}= \vec{i}-\vec{k},\vec{b}=x\vec{i}+\vec{j}+\brak{1-x}\vec{k}$ and $\vec{c}= y\vec{i}+x\vec{j}+\brak{1+x-y}\vec{k}$. Then $\sbrak{\vec{a} \ \vec{b} \ \vec{c}}$ depends on 
\hfill (2001S)
\begin{enumerate}
\begin{multicols}{4}
\item only $x$
\item only $y$
\item Neither $x$ Nor $y$
\item both $x$ and $y$
\end{multicols}
\end{enumerate}
\item If $\vec{a},\vec{b}$ and $\vec{c}$ are unit vectors, then \\
$\abs{\vec{a}-\vec{b}}^2+\abs{\vec{b}-\vec{c}}^2+\abs{\vec{a}-\vec{b}}^2$ does not exceed 
\hfill (2001S)
\begin{enumerate}
\begin{multicols}{4}
    \item $4$
    \item $9$
    \item $8$
    \item $6$
\end{multicols}
\end{enumerate}
\item If $\vec{a}$ and $\vec{b}$ are two unit vectors such that $\vec{a}+2\vec{b}$ and $5\vec{a}-4\vec{b}$ are perpendicular to each other then the angle between $\vec{a}$ and $\vec{b}$ is 
\hfill (2002S)
\begin{enumerate}
\begin{multicols}{2}
    \item $45\degree$
    \item $60\degree$
    \item $\arccos{\frac{1}{3}}$
    \item $\arccos{\frac{2}{7}}$
\end{multicols}
\end{enumerate}
\item Let $\vec{V}= 2\vec{i}+\vec{j}-\vec{k}$ and $\vec{W}= \vec{i}+3\vec{k}$. If $\vec{U}$ is a unit vector, then the maximum value of the scalar triple product $\brak{U \ V \ W}$ is 
\hfill (2002S)
\begin{enumerate}
\begin{multicols}{2}
    \item $-1$
    \item $\sqrt{10}+\sqrt{6}$
    \item $\sqrt{59}$
    \item $\sqrt{60}$
\end{multicols}
\end{enumerate}
\item The value of $k$ such that $\frac{x-4}{1}=\frac{y-2}{1}=\frac{z-k}{2}$ lies in the plane $2x-4y+z=7$, is
\hfill (2003S)
\begin{enumerate}
\begin{multicols}{2}
    \item $7$
    \item $-7$
    \item no real value
    \item $4$
\end{multicols}
\end{enumerate}
\item The value of $'a'$ so that the volume of parallelopiped formed by $\vec{i}+a\vec{j}+\vec{k},\vec{j}+a\vec{k}$ and $a\vec{i}+\vec{k}$ becomes minimum is 
\hfill (2003S)
\begin{enumerate}
\begin{multicols}{4}
    \item $-3$
    \item $3$
    \item $\frac{1}{\sqrt{3}}$
    \item $\sqrt{3}$
\end{multicols}
\end{enumerate}
\item If $\vec{a}=\vec{i}+\vec{j}+\vec{k}$, $\vec{a}.\vec{b}=1$ and $\vec{a} \times \vec{b} = \vec{j} - \vec{k}$. Then $\vec{b}$ is 
\hfill (2004S)
\begin{enumerate}
\begin{multicols}{2}
    \item $\vec{i}-\vec{j}+\vec{k}$
    \item $2\vec{j}-\vec{k}$
    \item $\vec{i}$
    \item $2\vec{i}$
\end{multicols}
\end{enumerate}
\item If the lines \\
$\frac{x-1}{2}=\frac{y+1}{3}=\frac{z-1}{4}$ and $\frac{x-3}{1}=\frac{y-k}{2}=\frac{z}{1}$ interest, then the value of $k$ is 
\hfill (2005S)
\begin{enumerate}
\begin{multicols}{4}
    \item $\frac{3}{2}$
    \item $\frac{9}{2}$
    \item $\frac{2}{9}$
    \item $\frac{-3}{2}$
\end{multicols}
\end{enumerate}


\end{enumerate}
\end{document}



























            
