%iffalse
\let\negmedspace\undefined
\let\negthickspace\undefined
\documentclass[journal,12pt,onecolumn]{IEEEtran}
\usepackage{cite}
\usepackage{amsmath,amssymb,amsfonts,amsthm}
\usepackage{algorithmic}
\usepackage{graphicx}
\usepackage{textcomp}
\usepackage{xcolor}
\usepackage{txfonts}
\usepackage{listings}
\usepackage{enumitem}
\usepackage{mathtools}
\usepackage{gensymb}
\usepackage{comment}
\usepackage[breaklinks=true]{hyperref}
\usepackage{tkz-euclide} 
\usepackage{listings}
\usepackage{gvv}                                        
%\def\inputGnumericTable{}                                 
\usepackage[latin1]{inputenc}                                
\usepackage{color}
\usepackage{array}                                        
\usepackage{longtable}                                    
\usepackage{calc}                                         
\usepackage{multirow}                                     
\usepackage{hhline}                                       
\usepackage{ifthen}                                       
\usepackage{lscape}
\usepackage{tabularx}
\usepackage{array}
\usepackage{float}
\usepackage{enumitem,multicol}

\newtheorem{theorem}{Theorem}[section]
\newtheorem{problem}{Problem}
\newtheorem{proposition}{Proposition}[section]
\newtheorem{lemma}{Lemma}[section]
\newtheorem{corollary}[theorem]{Corollary}
\newtheorem{example}{Example}[section]
\newtheorem{definition}[problem]{Definition}
\newcommand{\BEQA}{\begin{eqnarray}}
\newcommand{\EEQA}{\end{eqnarray}}
\newcommand{\define}{\stackrel{\triangle}{=}}
\theoremstyle{remark}
\newtheorem{rem}{Remark}

% Marks the beginning of the document
\begin{document}
\bibliographystyle{IEEEtran}
\vspace{3cm}

\title{2008 MA 52-68}
\author{EE24BTECH11051 - Prajwal}
\maketitle

\bigskip

\renewcommand{\thefigure}{\theenumi}
\renewcommand{\thetable}{\theenumi}


\begin{enumerate}
\item For two random variables $X$ and $Y$ , the regression lines are given by $Y=5X-15$ and $Y=10X-35$. Then regression coefficient of $X$ and $Y$ is
 \begin{enumerate}
 \begin{multicols}{4}
     \item $0.1$
     \item $0.2$
     \item $5$
     \item $10$
 \end{multicols}
\end{enumerate}


\item In an examination there are $80$ questions each having four choices. Exactly one of these is choices is correct abd the other three are wrong. A student is awarded $1$ mark for each correct answer, and $-0.25$ for each wrong answer. If a student ticks the answer of each question randomly, then the expected value of his/her total marks in the examination is 
\begin{enumerate}
 \begin{multicols}{4}
     \item $-15$
     \item $0$
     \item $5$
     \item $20$
 \end{multicols}
\end{enumerate}

\item Let $X_1,X_2,\dots,X_n$ be an random sample from uniform distribution on $\sbrak{0,\theta}$. Then the maximum likelihood estimator (MLE) of $\theta$ based on the above random sample is
\begin{enumerate}
 \begin{multicols}{2}
     \item $\frac{2}{n}\sum X_i$
     \item Min$\sbrak{X_1,X_2,\dots,X_n}$
     \item $\frac{1}{n}\sum X_i$
     \item Max$\sbrak{X_1,X_2,\dots,X_n}$
 \end{multicols}
\end{enumerate}

\item  the cost matrix of a transportation problem is given by \\

\begin{align}
\begin{tabular}{|c|c|c|c|}
\hline
1 & 2 & 3 & 4 \\
\hline
4 & 3 & 2 & 0 \\
\hline
0 & 2 & 2 & 1 \\
\hline
\end{tabular}    
\end{align}\\

The following are the values of the variables in a feasible solution.\\
$x_12=6,x_23=2,x_24=6,x_31=4,x_33=6$\\
then which of the following is correct?
\begin{enumerate}
 \begin{multicols}{2}
     \item The solution is degenerate and basic
     \item The solution is non-degenerate and basic
     \item The solution is degenerate and non-basic
     \item The solution is non-degenerate and non-basic
 \end{multicols}
\end{enumerate}


\item the maximum value of $z=3x_1+x_2$ subject to $2x_1+x_2\leq 3,x_1\leq3$ and $x_1,x_2\geq0$ is
\begin{enumerate}
 \begin{multicols}{4}
     \item $0$
     \item $4$
     \item $6$
     \item $9$
 \end{multicols}
\end{enumerate}

\item Consider the following maximizing problem $z=2x_1+3x_2-4x_3+x_4$ subject to 
\begin{align}
    x_1+x_2+ x_3=0 \\
  x_1+x_2-x_3=0 = 10 \\
   2x_1+3x_2-4x_3+x_4=0 \\
   x_1,x_2,x_3,x_4\geq0
\end{align}\\
Then
\begin{enumerate}
 \begin{multicols}{1}
     \item $\brak{1,0,1,4}$ is a basic feasible solution but $\brak{2,0,0,4}$ is not
     \item $\brak{1,0,1,4}$ is not a basic feasible solution but $\brak{2,0,0,4}$ is 
     \item neither $\brak{1,0,1,4}$ nor $\brak{2,0,0,4}$ is a basic feasible solution
     \item both of $\brak{1,0,1,4}$ and $\brak{2,0,0,4}$ are  basic feasible solutions
 \end{multicols}
\end{enumerate}

\item In the closed system of a simple harmonic motion of a  pendulum, let $H$ denote the Hamiltonian and $E$ be the total energy. Then
\begin{enumerate}
\begin{multicols}{2}
     \item $H$ is a constant and $H=E$
     \item $H$ is a constant and $H \neq E$
     \item $H$ is not constant and $H=E$
     \item $H$ is not constant and $H\neq E$
 \end{multicols}
\end{enumerate}

\item The possible values fo $\alpha$ for which the variational problem
\begin{align}
    J\sbrak{y(x)} = \int_{0}^{1} \brak{3y^2 + 2x^3 y'}dx, y(\alpha) = 1
\end{align}\\
has extremals are
\begin{enumerate}
 \begin{multicols}{4}
     \item $-1,0$
     \item $0,1$
     \item $-1,1$
     \item $-1,0,1$
 \end{multicols}
\end{enumerate}

\item  The functional $\int_{0}^{1} \brak{y'^2 + x^3}dx$,given $y(1)=1$,achieves its
\begin{enumerate}
 \begin{multicols}{1}
     \item weak maximum on all its external
     \item weak minimum on all its external
     \item weak minimum on some,but not on all of its externals
     \item weak maximum on some,but not on all of its externals
 \end{multicols}
\end{enumerate}

\item The integral equation\\
\begin{align}
    x(t) = \sin t + \lambda \int_{0}^{t} \brak{s^2 t^3 + e^{s^2 + t^3}} x(s)ds,0 \leq t \leq 1,\lambda \in \mathbb{R},\lambda \neq 0
\end{align}
\begin{enumerate}
 \begin{multicols}{1}
     \item all non-zero values of $\lambda$
     \item no value of $\lambda$
     \item only countably many positive values of $\lambda$
     \item only countably many negative values of $\lambda$
 \end{multicols}
\end{enumerate}

\item The integral equation $x(t) - \int_{0}^{1} \sbrak{\cos t \sec s \ x(s)}ds = \sin h t,0 \leq t \leq 1$,has
\begin{enumerate}
 \begin{multicols}{1}
     \item no solution
     \item a unique solution
     \item more than one but finitely many solution
     \item infinitely many solutions
 \end{multicols}
\end{enumerate}

\item If  $y_{i+1} = y_i + h \phi(f, x_i, y_i, h), i = 1, 2, \ldots$,  where \\
$\phi(f, x, y, h) = a f(x, y) + b f(x + h, y + h f(x, y))$,  is a second order accurate scheme to solve the
initial value problem  $\frac{dy}{dx} = f(x, y), y(x_0) = y_0$,  then  $a$  and  $b$, respectively, are
\begin{enumerate}
 \begin{multicols}{4}
     \item $\frac{h}{2},\frac{h}{2}$
     \item $1,-1$
     \item $\frac{1}{2},\frac{1}{2}$
     \item $h,-h$
 \end{multicols}
\end{enumerate}

\item If a quadrature formula $\frac{3}{2}f\brak{-\frac{1}{3}}+Kf\brak{\frac{1}{3}}+\frac{1}{2}f\brak{1}$,that approximates $\int_{-1}^{1}\brak{f\brak{x}}dx$,is found to be exact for quadratic polynomials,then the value of $K$ is
\begin{enumerate}
 \begin{multicols}{4}
     \item $2$
     \item $1$
     \item $0$
     \item $-1$
 \end{multicols}
\end{enumerate}

\item If $\begin{pmatrix}
1 & 4 & 3 \\
2 & 7 & 9 \\
5 & 8 & a
\end{pmatrix} = 
\begin{pmatrix}
l_{11} & 0 & 0 \\
l_{21} & l_{22} & 0 \\
l_{31} & l_{32} & -53
\end{pmatrix}
\begin{pmatrix}
1 & u_{12} & u_{13} \\
0 & 1 & u_{23} \\
0 & 0 & 1
\end{pmatrix}$, then the value of $a$ is 
\begin{enumerate}
 \begin{multicols}{4}
     \item $-2$
     \item $-1$
     \item $1$
     \item $2$
 \end{multicols}
\end{enumerate}

\item Using the least  squares method, if a curve $y=ax^2+bx+c$ is fitted to the collinear data points $\brak{-1,- 3},\brak{1,1},\brak{3, 5}$ and $\brak{7, 13}$,then the triplet $\brak{a,b,c}$ is equal to
\begin{enumerate}
 \begin{multicols}{4}
     \item $\brak{-1,2,0}$
     \item $\brak{0,2,-1}$
     \item $\brak{2,-1,0}$
     \item $\brak{0,-1,2}$
 \end{multicols}
\end{enumerate}

\item A quadratic polynomial $p(x)$ is constructed by interploting the data points $\brak{0,1},\brak{1,e}$ and $\bark{2, e^2}$.If $\sqrt{e}$ is approximated by  using $p(x)$, then its approximately value is 
\begin{enumerate}
 \begin{multicols}{4}
     \item $\frac{1}{8}\brak{3+6e-e^2}$
     \item $\frac{1}{8}\brak{3-6e+e^2}$
     \item $\frac{1}{8}\brak{3-6e-e^2}$
     \item $\frac{1}{8}\brak{3+6e-e^2}$
 \end{multicols}
\end{enumerate}

\item The characteristics curve of $2yu_x+\brak{2x+y^2}u_x=0$ passing through $\brak{0,0}$ is 
\begin{enumerate}
 \begin{multicols}{4}
     \item $y^2=2\brak{e^x+x-1}$
     \item $y^2=2\brak{e^x-x+1}$
     \item $y^2=2\brak{e^x-x-1}$
     \item $y^2=2\brak{e^x+x+1}$
 \end{multicols}
\end{enumerate}

\end{enumerate}

\end{document}







            
