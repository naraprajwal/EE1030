%iffalse
\let\negmedspace\undefined
\let\negthickspace\undefined
\documentclass[journal,12pt,onecolumn]{IEEEtran}
\usepackage{cite}
\usepackage{amsmath,amssymb,amsfonts,amsthm}
\usepackage{algorithmic}
\usepackage{graphicx}
\usepackage{textcomp}
\usepackage{xcolor}
\usepackage{txfonts}
\usepackage{listings}
\usepackage{enumitem}
\usepackage{mathtools}
\usepackage{gensymb}
\usepackage{comment}
\usepackage[breaklinks=true]{hyperref}
\usepackage{tkz-euclide} 
\usepackage{listings}
\usepackage{gvv}                                        
%\def\inputGnumericTable{}                                 
\usepackage[latin1]{inputenc}                                
\usepackage{color}
\usepackage{array}                                        
\usepackage{longtable}                                    
\usepackage{calc}                                         
\usepackage{multirow}                                     
\usepackage{hhline}                                       
\usepackage{ifthen}                                       
\usepackage{lscape}
\usepackage{tabularx}
\usepackage{array}
\usepackage{float}
\usepackage{tikz}
\usepackage{enumitem,multicol}

\newtheorem{theorem}{Theorem}[section]
\newtheorem{problem}{Problem}
\newtheorem{proposition}{Proposition}[section]
\newtheorem{lemma}{Lemma}[section]
\newtheorem{corollary}[theorem]{Corollary}
\newtheorem{example}{Example}[section]
\newtheorem{definition}[problem]{Definition}
\newcommand{\BEQA}{\begin{eqnarray}}
\newcommand{\EEQA}{\end{eqnarray}}
\newcommand{\define}{\stackrel{\triangle}{=}}
\theoremstyle{remark}
\newtheorem{rem}{Remark}

% Marks the beginning of the document
\begin{document}
\bibliographystyle{IEEEtran}
\vspace{3cm}

\title{2011 AE 27-39}
\author{EE24BTECH11051 - Prajwal}
\maketitle

\bigskip

\renewcommand{\thefigure}{\theenumi}
\renewcommand{\thetable}{\theenumi}


\begin{enumerate}
\item Consider a beam in bending with a solid circular cross-section of $1mm^2$, which is subjected to a transverse shear force of $1N$. The shear stress at the center of the cross-section (in $N/mm^2$) is $\dots$
 
\item A simply supported slender column of square cross section (width=depth=$d$) has to be designed such that it buckles at the same instant as it yields. Length of the column is given to be $1.57m$ and
it is made of a material whose Young’s modulus is $200GPa$ and yield stress is $240MPa$. The width, $d$, of the column (in $cm$) should be $\dots$

\item A turbojet powered aircraft is flying at Mach number 0.8 at an altitude of $10km$. The inlet and exit areas of the engine are $0.7m^2$ and $0.4m^2$ respectively. The exhaust gases have velocity of $500 m/s$ and pressure of $60kPa$. The free stream pressure, density and speed of sound are $26.5kPa$, $0.413kg/m^3$ and $299.5m/s$ respectively. The thrust of the engine (in kN) is $\dots$

\item  A low speed wind tunnel has a contraction ratio of $14:1$ and the cross-sectional area of the test section is $1m^2$. The static pressure difference between the settling chamber and the test section is
$40cm$ of water column. Assume $g=9.81m/s^2$,$\rho_{air}=1.2kg/m^3$ and $\rho_{water}=1000kg/m^3$. The speed of air in the test section (in $m/s$) is
$\dots$
\\ \\ 
Question Q.31 to Q.55 multiple choice type
\item Consider the function $f(x)=x-\sin x$. The Newton-Raphson iteration formula to find the root of the function starting from an initial guess at $x^{\brak{0}}$ iteration $k$ is
\begin{enumerate}
 \begin{multicols}{2}
     \item $x^{(k+1)} = \frac{\sin x^{\brak{k}} - x^{\brak{k}} \cos x^{\brak{k}}}{1 - \cos x^{\brak{k}}}$
     \item $x^{(k+1)} = \frac{\sin x^{\brak{k}} - x^{\brak{k}} \cos x^{\brak{k}}}{1 + \cos x^{\brak{k}}}$
     \item $x^{(k+1)} = \frac{\sin x^{\brak{k}} + x^{\brak{k}} \cos x^{\brak{k}}}{1 - \cos x^{\brak{k}}}$
     \item $x^{(k+1)} = \frac{\sin x^{\brak{k}} + x^{\brak{k}} \cos x^{\brak{k}}}{1 + \cos x^{\brak{k}}}$
 \end{multicols}
\end{enumerate}

\item Consider the matrix 
$\begin{bmatrix}
2 & a \\
b & 2  
\end{bmatrix}$ where a and b are real numbers. The two eigenvalues of this matrix $\lambda_1$ and $\lambda_2$ are real and distinct $\brak{\lambda_1 \neq \lambda_2}$ when 
\begin{enumerate}
 \begin{multicols}{4}
     \item $a<0$ and $b>0$
     \item $a>0$ and $b<0$
     \item $a<0$ and $b<0$
     \item $a=0$ and $b=0$
 \end{multicols}
\end{enumerate}

\item the solution of $\frac{dy}{dt}=y^3e^tt^2$ with initial condition $y(0)=1$ is given by
\begin{enumerate}
\begin{multicols}{2}
     \item $\frac{1}{9}e^t\brak{t+3}^2$
     \item $\sqrt{\frac{9}{5+2e^t\brak{t^2-2t+2}}}$
     \item $\frac{4e^t}{\brak{t+2}^2}$
     \item $\sqrt{\frac{1}{5-2e^t\brak{t^2-2t+2}}}$
 \end{multicols}
\end{enumerate}

\item A jet engine is operating at a Mach number of $0.8$ at an altitude of $10 km$. The efficiency of the air intake is $0.8$ and that of the compressor is $0.87$. The total temperatures (in K) at the exits of the air intake and the compressor respectively are\\
	(Amibient presure=26.5kPa;Ambient temperature =223.3K;Gas constant,\gamma=1.4;$p_{rc}$=8)
\begin{enumerate}
 \begin{multicols}{4}
     \item 251.9 and 458.2
     \item 234.9 and 486.8
     \item 252.9 and 486.8
     \item 234.9 and 458.2
 \end{multicols}
\end{enumerate}

\item  A rocket engine is tested on a test bed under the ideal condition of fully expanded jet. The exhaust velocity is 2 km/s through a nozzle of area 2.5 m$^2$. The mass flow rate is 200 kg/s. The specific
impulse of the propellant and the thrust developed respectively are (assume g = 9.81 m/s$^2$)
\begin{enumerate}
 \begin{multicols}{2}
     \item 175.87 s and 200 kN
     \item 203.87 s and 400 kN
     \item 231.87 s and 200 kN
     \item 280.87 s and 400 kN
 \end{multicols}
\end{enumerate}

\item A body undergoes deformation under plane strain conditions when subjected to the following
	stresses (in MPa): $\rho_{xx}=450,\rho_{yy}=450,\tau_{xy}=75,\tau_{xz}=0,\tau_{yz}=0$.What are the remaining components of stresses (in MPa) and strains? Assume the material to be isotropic and linear-elastic $v=\frac{1}{3}$
\begin{enumerate}
 \begin{multicols}{1}
 \item $\sigma_{zz}=0,\epsilon_{xx}=0.00225,\epsilon_{xx}=0.00225,\gamma_{xy}=0.002,\gamma_{xz}=0,\gamma_{yz}=0$
     \item $\sigma_{zz} = 300,\epsilon_{xx} = 0.001,\epsilon_{yy} = 0.001,\gamma_{xy} = 0.001,\gamma_{xz} = 0,\gamma_{yz} = 0$
     \item $\sigma_{zz} = 300,\epsilon_{xx} = 0.00225,\epsilon_{yy} = 0.00225,\gamma_{xy} = 0.001,\gamma_{xz} = 0, \ \gamma_{yz} = 0$
     \item $\sigma_{zz} = 0,\epsilon_{xx} = 0.001,\epsilon_{yy} = 0.001,\gamma_{xy} = 0.002,\gamma_{xz} = 0, \ \gamma_{yz} = 0$
 \end{multicols}
\end{enumerate}

\item Which of the following Airy’s stress functions could satisfy the given boundary conditions,assuming constant values of $\sigma_{xx}=P,\sigma_{yy}=Q$ and $\tau_{xy}=R$,along tha boundary?

\centering

\begin{tikzpicture}
\tikzstyle{every node}=[font=\LARGE]
\draw (3,11.75) to[short] (9.25,11.75);
\draw (9.25,11.75) to[short] (9.25,9.25);
\draw (3,11.75) to[short] (3,9);
\draw (3,9) to[short] (9.25,9);
\draw (9.25,9) to[short] (9.25,9.5);
\draw [->, >=Stealth] (3.25,11.75) -- (3.25,12.75);
\draw [->, >=Stealth] (4,11.75) -- (4,12.75);
\draw [->, >=Stealth] (4.75,11.75) -- (4.75,12.75);
\draw [->, >=Stealth] (5.5,11.75) -- (5.5,12.75);
\draw [->, >=Stealth] (6.25,11.75) -- (6.25,12.75);
\draw [->, >=Stealth] (7,11.75) -- (7,12.75);
\draw [->, >=Stealth] (7.75,11.75) -- (7.75,12.75);
\draw [->, >=Stealth] (8.5,11.75) -- (8.5,12.75);
\draw [->, >=Stealth] (3,12.25) -- (3.75,12.25);
\draw [->, >=Stealth] (4.5,12.25) -- (5.5,12.25);
\draw [->, >=Stealth] (6,12.25) -- (7,12.25);
\draw [->, >=Stealth] (7.75,12.25) -- (8.5,12.25);
\draw [->, >=Stealth] (9.25,11.5) -- (10.25,11.5);
\draw [->, >=Stealth] (9.25,10.75) -- (10.25,10.75);
\draw [->, >=Stealth] (9.25,10) -- (10.25,10);
\draw [->, >=Stealth] (9.25,9.25) -- (10.25,9.25);
\draw [->, >=Stealth] (8.75,9) -- (8.75,8);
\draw [->, >=Stealth] (7.75,9) -- (7.75,8);
\draw [->, >=Stealth] (6.75,9) -- (6.75,8);
\draw [->, >=Stealth] (5.75,9) -- (5.75,8);
\draw [->, >=Stealth] (4.75,9) -- (4.75,8);
\draw [->, >=Stealth] (3.75,9) -- (3.75,8);
\draw [->, >=Stealth] (3,9) -- (3,8);
\draw [->, >=Stealth] (3,9.5) -- (2,9.5);
\draw [->, >=Stealth] (3,10.25) -- (2,10.25);
\draw [->, >=Stealth] (3,11) -- (2,11);
\draw [->, >=Stealth] (3,11.5) -- (2,11.5);
\draw [->, >=Stealth] (2.5,11.75) -- (2.5,10.75);
\draw [->, >=Stealth] (2.5,10.5) -- (2.5,9.25);
\draw [->, >=Stealth] (9.5,9.25) -- (9.5,10.25);
\draw [->, >=Stealth] (9.5,10.75) -- (9.5,11.75);
\draw [->, >=Stealth] (4,8.75) -- (2.75,8.75);
\draw [->, >=Stealth] (5.5,8.75) -- (4.75,8.75);
\draw [->, >=Stealth] (7,8.75) -- (6.25,8.75);
\draw [->, >=Stealth] (8.5,8.75) -- (7.5,8.75);
\end{tikzpicture}
\begin{enumerate}
 \begin{multicols}{2}
     \item $\phi=P\frac{x^2}{2}+Q\frac{y^2}{2}-Rxy$
     \item $\phi=P\frac{y^2}{2}+Q\frac{x^2}{2}+Rxy$
     \item $\phi=P\frac{y^2}{2}+Q\frac{x^2}{2}-Rxy$
     \item $\phi=P\frac{x^2}{2}+Q\frac{y^2}{2}+Rxy$
 \end{multicols}
\end{enumerate}

\item An aircraft is performing a coordinated turn manoeuvre at a bank angle of 30o and forward speed of
100 m/s. Assume g = 9.81ms$^-2$. The load factor and turn radius respectively are
\begin{enumerate}
 \begin{multicols}{1}
     \item $(\frac{2}{\sqrt{3}})$ and 1.76 km
     \item $\sqrt{3}$ and 17.6 km
     \item $2$ and $0.18$ km
     \item $\brak{\frac{2}{\sqrt{3}}}$ and 17.6 km
 \end{multicols}
\end{enumerate}

\item An aircraft in a steady level flight at forward speed of 50 m/s suddenly rolls by 180o and becomes
inverted. If no other changes are made to the configuration or controls of the aircraft, the nature of
the subsequent flight path taken by the aircraft and its characteristic parameter(s) (assume g = 9.81
ms$^{-2}$) are
\begin{enumerate}
 
     \item straight line path with a speed of 50 m/s
     \item upward circular path with a speed of 50 m/s and radius of            		   127.4 m
     \item downward circular path with a speed of 50 m/s and radius of 				   127.4 m/s
     \item downward circular path with a speed of 25 m/s and radius of       	       254.8 m/s
\end{enumerate}


\end{enumerate}

\end{document}


