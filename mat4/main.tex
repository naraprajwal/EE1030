%iffalse
\let\negmedspace\undefined
\let\negthickspace\undefined
\documentclass[journal,12pt,onecolumn]{IEEEtran}
\usepackage{cite}
\usepackage{amsmath,amssymb,amsfonts,amsthm}
\usepackage{algorithmic}
\usepackage{graphicx}
\usepackage{textcomp}
\usepackage{xcolor}
\usepackage{txfonts}
\usepackage{listings}
\usepackage{enumitem}
\usepackage{mathtools}
\usepackage{gensymb}
\usepackage{comment}
\usepackage[breaklinks=true]{hyperref}
\usepackage{tkz-euclide} 
\usepackage{listings}
\usepackage{gvv}                                        
%\def\inputGnumericTable{}                                 
\usepackage[latin1]{inputenc}                                
\usepackage{color}
\usepackage{array}                                        
\usepackage{longtable}                                    
\usepackage{calc}                                         
\usepackage{multirow}                                     
\usepackage{hhline}                                       
\usepackage{ifthen}                                       
\usepackage{lscape}
\usepackage{tabularx}
\usepackage{array}
\usepackage{float}
\usepackage{enumitem,multicol}

\newtheorem{theorem}{Theorem}[section]
\newtheorem{problem}{Problem}
\newtheorem{proposition}{Proposition}[section]
\newtheorem{lemma}{Lemma}[section]
\newtheorem{corollary}[theorem]{Corollary}

\newtheorem{example}{Example}[section]
\newtheorem{definition}[problem]{Definition}
\newcommand{\BEQA}{\begin{eqnarray}}
\newcommand{\EEQA}{\end{eqnarray}}
\newcommand{\define}{\stackrel{\triangle}{=}}


% Marks the beginning of the document
\begin{document}
\bibliographystyle{IEEEtran}
\vspace{3cm}

\title{Jan 7 S2 16-30}
\author{EE24BTECH11051 - Prajwal}
\maketitle

\bigskip

\renewcommand{\thefigure}{\theenumi}
\renewcommand{\thetable}{\theenumi}


\begin{enumerate}


\item Let $\alpha$ and $\beta$ are the roots of the equation $x^2-x-1=0$. If $p_k = (\alpha)k + (\beta)k, k\geq1$ then which one of the following statements is not true?
 \begin{enumerate}
 \begin{multicols}{1}
     \item $(p_1 + p_2 + p_3 + p_4 + p_5) = 26$
     \item $ p_5 = 11$
     \item $p_5 = p_2 * p_3$
     \item $p_3=p_5-p_4$
 \end{multicols}
\end{enumerate}


\item The area of the region \{$(x, y)\in R | 4x^2 \leq y \leq 8x + 12$\} is :
\begin{enumerate}
 \begin{multicols}{1}
     \item $\frac{125}{3}$
     \item $\frac{128}{3}$
     \item $\frac{124}{3}$
     \item $\frac{127}{3}$
 \end{multicols}
\end{enumerate}

\item The value of $c$ in Lagrange's mean value theorem for the function $f(x) = x^3-4x^2+8x+11$, where $x \in \sbrak{0,1}$ is
\begin{enumerate}
 \begin{multicols}{1}
     \item $\frac{\brak{4-\sqrt{7}}}{3}$
     \item $\frac{2}{3}$
     \item $\frac{\brak{\sqrt{7}-2}}{3}$
     \item $\frac{\brak{4-\sqrt{5}}}{3}$
 \end{multicols}
\end{enumerate}

\item  Let $y = y(x)$ be a function of $x$ satisfying $y\brak{\sqrt{1-x^2}}=k-x\brak{\sqrt{1-y^2}}$ where $k$ is a constant and $y(1/2) = -1/4$. Then $\frac{dy}{dx}$ at $x=\frac{1}{2}$ is equal to:
\begin{enumerate}
 \begin{multicols}{1}
     \item $\frac{-\sqrt{5}}{2}$
     \item $\frac{\sqrt{5}}{2}$
     \item $\frac{-\sqrt{5}}{4}$
     \item $\frac{2}{\sqrt{5}}$
 \end{multicols}
\end{enumerate}


\item Let the tangents drawn from the origin to the circle, $x^2 + y^2 - 8x - 4y + 16 = 0$ touch it at the points $A$ and $B$. The $\brak{AB}^2$ is equal to
\begin{enumerate}
 \begin{multicols}{1}
     \item $\frac{32}{5}$
     \item $\frac{64}{5}$
     \item $\frac{52}{5}$
     \item $\frac{56}{5}$
 \end{multicols}
\end{enumerate}

\item If system of linear equations \\
\begin{align*}
   x+y+z=6  \\
   x + 2y + 3z = 10 \\
   3x + 2y + \lambda z = \mu
\end{align*}
has more than two solutions, then $\mu - \lambda^2$ is equal to


\item If the function $f$ defined on $\brak{\frac{-1}{3},\frac{1}{3}}$ by
    $f(x) = $
$\begin{cases}
   \brak{\frac{1}{x}}\log\brak{\frac{1+3x}{1-2x}}  & \text{if} x \neq 0\\
    k & \text{if} \  x=0
\end{cases}$\\ 
is continuous, then $k$ is equal to

\item If the foot of perpendicular drawn from the point $\brak{1,0,3}$ on a line passing through $\brak{\alpha,7,1}$ is $\brak{\frac{5}{3},\frac{7}{3},\frac{17}{3}}$ then $\alpha$ is equal to:

\item  If the mean and variance of eight numbers $3, 7, 9, 12, 13, 20, x$ and $y$ be $10$ and $25$ respectively then $xy$ is equal to

\item Let $X = \{n \in N: 1 \leq n \leq 50\}$. If $A= \{n \in N: n \ \text{is a multiple of} \ 2\}$ and $B = \{n \in N: n \ \text{is a multiple of} \ 7\}$, then the number of elements in the smallest subset of $X$ containing both $A$ and $B$ is .









\end{enumerate}

\end{document}







            
