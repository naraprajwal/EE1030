\let\negmedspace\undefined
\let\negthickspace\undefined
\documentclass[journal]{IEEEtran}
\usepackage[a5paper, margin=10mm, onecolumn]{geometry}
%\usepackage{lmodern} % Ensure lmodern is loaded for pdflatex
\usepackage{tfrupee} % Include tfrupee package
\setlength{\headheight}{1cm} % Set the height of the header box
\setlength{\headsep}{0mm}     % Set the distance between the header box and the top of the text
\usepackage{gvv-book}
\usepackage{gvv}
\usepackage{cite}
\usepackage{amsmath,amssymb,amsfonts,amsthm}
\usepackage{algorithmic}
\usepackage{graphicx}
\usepackage{textcomp}
\usepackage{xcolor}
\usepackage{txfonts}
\usepackage{listings}
\usepackage{enumitem}
\usepackage{mathtools}
\usepackage{gensymb}
\usepackage{comment}
\usepackage[breaklinks=true]{hyperref}
\usepackage{tkz-euclide} 
\usepackage{listings}
% \usepackage{gvv}                                        
\def\inputGnumericTable{}                                 
\usepackage[latin1]{inputenc}                                
\usepackage{color}                                            
\usepackage{array}                                            
\usepackage{longtable}                                       
\usepackage{calc}                                             
\usepackage{multirow}                                         
\usepackage{hhline}                                           
\usepackage{ifthen}                                           
\usepackage{lscape}
\renewcommand{\thefigure}{\theenumi}
\renewcommand{\thetable}{\theenumi}
\setlength{\intextsep}{10pt} % Space between text and floats
\numberwithin{equation}{enumi}
\numberwithin{figure}{enumi}
\renewcommand{\thetable}{\theenumi}
\begin{document}
\bibliographystyle{IEEEtran}
\title{Matrix}
\author{EE24BTECH11051 - Prajwal}
% \maketitle
% \newpage
% \bigskip
{\let\newpage\relax\maketitle}
\section{1.5.9}
Find the ratio in which the $Y$ axis divides the line segment joining the points $\vec{A}\brak{5,-6}$ and $\vec{B}\brak{-1,-4}$. Also, find the coordinates of the point of intersection.\\ 
Solution:-\\
Section Formulae,\hfill (eq 1)
 \begin{align*}
 C=\frac{A+kB}{1+k} 
 \end{align*}
 where $k$ is the ratio in which $C$ divides $A$ and $B$,
 If $C$ is on the $Y$ axis then,
 \begin{align*}
     C = \myvec{0 \\ y}
 \end{align*}
 Given,
 \begin{align*}
 A = \myvec{5 \\ -6},
 B = \myvec{-1 \\ -4}
 \end{align*}
 Putting the value of $A,B$ and $C$ in equation 1,
 \begin{align*}
 C=\frac{A+kB}{1+k}\\
 \myvec{0 \\ y} =
 \frac{\myvec{5 \\ -6} + k\myvec{-1 \\ -4}}{1+k}\\
 \implies \myvec{0 \\ y} =
 \frac{\myvec{5-k \\ -6-4k}}{1+k}\\
 \implies \myvec{0 \\ y}(1+k) = 
 \myvec{5-k \\ -6-4k}\\
 \implies \myvec{0 \\ y(1+k)} =
 \myvec{5-k \\ -6-4k}
 \end{align*}
 Comparing matrix in both side,
 \begin{align*}
 5-k=0\\
 \implies k=5
 \end{align*}
 and,
 \begin{align*}
     y(1+k)=-6-4k
 \end{align*}
 putting the value of k,
 \begin{align*}
y(1+k)=-6-4k\\
 \implies y(1+5)=-6-4(5)\\
 \implies y(6)=-26\\
 \implies y=\frac{-26}{6}\\
 \implies y=\frac{-13}{3}
 \end{align*}
 Hence, $Y$ axis divides the line segment $AB$ in $5:1$ ratio
 And,
 The point on $Y$ axis is $\vec{C}\brak{0,\frac{-13}{3}}$
\end{document}